\documentclass{article}

\usepackage{hyperref}

\usepackage{setspace}
\doublespacing

\usepackage{fancyhdr}
\pagestyle{fancy}
\fancyhf{}

\lhead{Mathew Zaleski}
\chead{zaleskim}
\rhead{L1 BA4222}
\rfoot{Page \thepage}

\begin{document}

\title{CSC300 template assignment using BibTeX to do dirty work for bibliographies}
\author{Mathew Zaleski}
\date{2017-11-1}

\maketitle
\thispagestyle{fancy}
\section{Introduction}

Bibtex \cite{original-bibtex,bibtex} is a program that works in
conjunction with latex to convert
a text database of references into a bibliography. 

This document illustrates the use of BibTeX\@.  It follows  the webpage
\cite{bibtex-tutorial}  but replaces the references of that tutorial
with a couple of the required readings for csc300, etc. \footnote{You can also clone it from github: \url{https://github.com/jmzaleski/csc300-latex-bibtex-ieee-template}
or from the sharelatex website at \cite{shared-latex-page}}


By way of demo, You may want to refer to a journal article \cite{warren-brandeis-1890}
or a conference proceeding \cite{siva05} or a \cite{gosling-interview}
or even a book \cite {java-jvm-spec}!

\section{Thesis}

Although it has a steep, nasty learning curve, bibtex and latex are
worth considering because, at least in some academic environments, a
life sentence of reformatting citations and bibliographies is even
worse.


\newpage

just making sure headers are working on the new page..

\bibliographystyle{IEEEtran}

\bibliography{my-bibtex-file}

\end{document}
